\documentclass{article}
\usepackage{amsmath}
\usepackage{amsthm}
\usepackage{amssymb}
% Theorem Styles
\newtheorem*{theorem}{Theorem}
\begin{document}
\begin{theorem}[Glivenko-Cantelli]
Let \( F_n \) be the empirical distribution function of a sample of size \( n \),
 for \((X_i)_{i\ge1} \in \mathbf{R}\) i.i.d random variables with distribution function \(F\). Then

\[ \lim_{n \to \infty} \sup_{x \in \mathbf{R}} |F_n(x) - F(x)| = 0 \quad \text{a.s.} \]
\end{theorem}
\begin{proof} 
Let \(t_1,\ldots,t_k \in \mathbf{R}\). Then since \(\forall j\le k \quad (\mathbf{1}_{(-\infty,t_j]}(X_i ))_{i\ge1}\) are also i.i.d random variables.
By the strong law of large numbers we know that 
\begin{equation}
    \label{eq1}
    \forall j \le k: \quad |F_n(t_j)-F(t_j)| \to 0 \quad (a.s.).
\end{equation}
So for every \(j\le k\) we have a set \(A_j \in \Omega\) with \(\mathbf{P}(A_j) = 1\) such that \eqref{eq1} holds.
Clearly \(\mathbf{P}(\bigcap_{j\le k}A_j) = 1\), so (by choosing the largest \(N_j\) in the definition of the limits) we have:
\begin{equation}
    \label{eq2}
    \max_{j= 1,\ldots,k} |F_n(t_j)-F(t_j)| \to 0 \quad (a.s.).
\end{equation}
Now, let \(h \nearrow t_j\). Then
\begin{align*}
    F_n(t_j^-) & = \lim_{h\nearrow t_j} \frac{1}{n} \sum_{i = 1,\ldots,n} \mathbf{1}_{(-\infty,h]}(X_i)\\
    & = \frac{1}{n} \sum_{i = 1,\ldots,n} \mathbf{1}_{(-\infty,t_j)}(X_i)\\
\end{align*}
Because \((\mathbf{1}_{(-\infty,t_j)}(X_i))_{i\ge 1}\) are i.i.d random variables (with finite expectation), the strong law of large numbers gives us:
\begin{equation*}
    |F_n(t_j^-) - F(t_j^-))| = |F_n(t_j^-) - \mathbf{P}(X_m \in (-\infty,t_j))| \to 0 \quad(a.s.).
\end{equation*}
Using the same argument as before we can now conclude:
\begin{equation}
    \label{eq3}
    \max_{j= 1,\ldots,k} |F_n(t_j^-)-F(t_j^-)| \to 0 \quad (a.s.).
\end{equation}
Continuing, fix any \(\varepsilon>0\) and choose \(t_j = \inf\{t \in \mathbf{R} : F(t) \ge j\varepsilon\}\) for \(i= 1,\ldots, \left\lfloor \frac{1}{\varepsilon} \right\rfloor \). (Note that \(t_0 = -\infty\)).
Then \(\forall t \in \mathbf{R}\) there is a \(j \in \mathbf{N}\) with \(t \in (t_{j-1},t_{j})\) since \(F\) is a cdf and \(j\varepsilon \le 1\).
Now we estimate using \(t < t_i\) and \(t> t_{i-1}\):
\begin{align*}
    F_n(t)-F(t) &\le F_n(t_j^-) -F(t_{j-1}) \\
                &\le (F_n(t_j^-) -F(t_j^-)) +(F(t_j^-)- F_n(t_{j-1})) +(F_n(t_{j-1})-F(t_{j-1}) ) \\
                &\le \max_{j=1,\ldots,k}|F_n(t_j^-) -F(t_j^-)| +(F(t_j^-)- F_n(t_{j-1})) + \max_{j=1,\ldots,k}|F_n(t_j)-F(t_j)|\\
                &\le \max_{j=1,\ldots,k}|F_n(t_j^-) -F(t_j^-)| +j\varepsilon- F_n(t_{j-1}) + \max_{j=1,\ldots,k}|F_n(t_j)-F(t_j)|\\
\end{align*} 
By \eqref{eq1} we have for all \(n \ge N(\varepsilon)\), that \(F(t_{j-1})-F_n(t_{j-1}) \le \varepsilon \iff -F_n(t_{j-1}) \le \varepsilon-F(t_{j-1}) \le \varepsilon - (j-1)\varepsilon  \), almost surely.
We conclude, that 
\[ 
    F_n(t)-F(t) \le \max_{j=1,\ldots,k}|F_n(t_j^-) -F(t_j^-)| +2\varepsilon + \max_{j=1,\ldots,k}|F_n(t_j)-F(t_j)| \quad (a.s.).
\]
A completely symmetrical argument then shows: 
\begin{equation}
    \label{eq4}
    |F_n(t)-F(t)| \le \max_{j=1,\ldots,k}|F_n(t_j^-) -F(t_j^-)| +2\varepsilon + \max_{j=1,\ldots,k}|F_n(t_j)-F(t_j)| \quad (a.s.).
\end{equation}
Note that the choice of the numbers \(t_j\) only depends on \(\varepsilon\).
Therefore for \(\varepsilon > 0\) \eqref{eq4} implies 
\[
    \sup_{t\in \mathbf{R}}|F_n(t)-F(t)| \le \max_{j=1,\ldots,k}|F_n(t_j^-) -F(t_j^-)| +2\varepsilon + \max_{j=1,\ldots,k}|F_n(t_j)-F(t_j)| \quad (a.s.).
\]
For N large enough. Combining this with \eqref{eq2} and \eqref{eq3} and choosing \(N_1 \ge N\) large enough (also greater than the N necessary in \eqref{eq2} and \eqref{eq3}), yields:
\[ 
    \sup_{t\in \mathbf{R}}|F_n(t)-F(t)| \le \varepsilon  +2\varepsilon + \varepsilon  \text{ for } n \ge N_1 \quad (a.s.).
    \]
    Therefore we also have
    \[ 
        \limsup_{n\to \infty}\sup_{t\in \mathbf{R}}|F_n(t)-F(t)| \le +5\varepsilon  \quad (a.s.).
    \]
    Because \(\varepsilon > 0 \)  was arbitrary, we conclude
    \[
    0 \le \liminf_{n\to \infty}\sup_{t\in \mathbf{R}}|F_n(t)-F(t)| \le \limsup_{n\to \infty}\sup_{t\in \mathbf{R}}|F_n(t)-F(t)|=0  \quad (a.s.),
    \]
    which yields the claim.
\end{proof}


\end{document}


